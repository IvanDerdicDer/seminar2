\chapter{Tehnologija} \label{chapter:tehnologija}
U ovom poglavlju se obrađuju tehnologije koje se koriste za ostvarivanje slojeva
jezerskog skladišta podataka u ovom radu. Jezersko skladište podataka ne
definira tehnologije koje se koriste za ostvarivanje slojeva, već samo slojeve i
njihove funkcionalnosti. U ovom poglavlju opisane su sljedeće tehnologije:
\begin{itemize}
    \item Apache Spark,
    \item Delta Lake.    
\end{itemize}

\section{Spark} \label{section:spark}
Spark je distribuirani okvir za obradu podataka otvorenog koda koji je
dizajniran za brzu i jednostavnu obradu velikih količina podataka. Spark
omogućava programerima da razviju složene aplikacije za obradu podataka i
analizu u nekoliko programskih jezika, uključujući Java, Python, Scala i R.

Spark je popularan zbog svoje sposobnosti da brzo i jednostavno obradi velike
količine podataka izvršavanjem na grozdu. Osim toga, Spark pruža mnoge
biblioteke i alate za obradu podataka, uključujući Spark SQL, Spark Streaming,
MLlib i GraphX.

Spark se često koristi u industriji za obradu podataka u stvarnom vremenu,
strojno učenje, obradu teksta, obradu slika i druga područja primjene. Spark je
postao popularan zbog svoje brzine obrade podataka, skalabilnosti i
fleksibilnosti u korištenju. 

Spark se u jezerskom skladištu podataka može koristiti za ostvarivanje sloja
unosa podataka (vidi poglavlje~(\ref{section:sloj_unosa_podataka})) i sloja
obrade podataka (vidi poglavlje~(\ref{section:sloj_obrade_podataka})). Za
detaljniji opis Sparka vidjeti poglavlje 1 iz \cite{spark2020}.

\section{Delta Lake} \label{section:delta_lake}
Delta Lake je projekt otvorenog koda koji se temelji na Apache Sparku, a
namijenjen je upravljanju i obradi podataka u velikim i složenim analitičkim
aplikacijama. Delta Lake kombinira karakteristike jezera podataka i skladišta
podataka u jednoj platformi koja je skalabilna, otporna na
pogreške i sposobna za rad u realnom vremenu.

Delta Lake omogućuje pohranu podataka u obliku tablica, s podrškom za
transakcije i verzioniranje. To omogućuje korisnicima da jednostavno dodaju,
ažuriraju ili brišu podatke, dok se istovremeno održava povijest promjena.

Delta Lake također pruža podršku za naprednu obradu podataka, poput upravljanja
s vremenom, verzioniranja shema i upravljanja sinkronizacijom podataka. Ove
značajke olakšavaju integraciju s drugim alatima i aplikacijama, što je korisno
u velikim poduzećima s kompleksnim IT okruženjima.

Delta Lake se često koristi u poslovima koji zahtijevaju brzu obradu podataka u
stvarnom vremenu i velike količine podataka, poput bankarstva, telekomunikacija,
e-trgovine i drugih industrija koje se bave velikim količinama podataka.

Delta Lake je tehnologija za upravljanje i obradu podataka koja se može
koristiti u sklopu jezerskog skladišta podataka. Može se koristiti za
ostvarivanje sloja pohrane podataka (vidi
poglavlje~(\ref{section:sloj_pohrane_podataka})). Za detaljniji opis Delta
Lake-a vidjeti
\cite{deltalake2023}
