\chapter{Teorija} \label{chapter:teorija}
U ovom poglavlju se obrađuje teorija Jezerskog Skladišta Podataka. U ovom
poglavlju dana je definicija Jezerskog Skladišta Podataka, te su opisani
sljedeći slojevi:
\begin{itemize}
    \item Sloj Unosa Podataka,
    \item Sloj Pohrane Podataka,
    \item Sloj Obrade Podataka.
\end{itemize}

\section{Jezersko Skladište Podataka} \label{section:jezersko_skladiste_podataka}
Jezersko Skladište Podataka (eng. Data Lakehouse) je arhitektura skladištenja
podataka koja kombinira karakteristike Jezera Podataka (eng. Data Lake) i
Skladišta Podataka (eng. Data Warehousea). To je centralizirano skladište
podataka koje omogućava analizu velike količine strukturiranih i
nestrukturiranih vrsta podataka u realnom vremenu ili u kasnijem trenutku.
Jezersko skladište podataka omogućuje integraciju podataka iz različitih izvora,
olakšava upravljanje podacima, smanjuje troškove i vrijeme potrebno za pripremu
podataka za analizu. Ova arhitektura skladištenja podataka postaje sve
popularnija u posljednje vrijeme jer olakšava analizu podataka, izvještavanje i
donošenje odluka u stvarnom vremenu. Za detaljniji opis Jezerskog Skladišta
Podataka vidjeti \cite[c.2]{datalakehouse2022}.

\section{Sloj Unosa Podataka} \label{section:sloj_unosa_podataka}
Sloj Unosa Podataka (engl. Data Ingestion Layer) u Jezerskom Skladištu Podataka
je sloj koji se koristi za prikupljanje i spremanje podataka iz različitih
izvora u Jezersko Skladište Podataka. Ovaj sloj obuhvaća dva načina prikupljanja
podataka:
\begin{enumerate}
    \item serijsko prikupljanje podataka,
    \item strujno prikupljanje podataka.
\end{enumerate}

Sloj Unosa Podataka omogućuje podatkovnim inženjerima da učinkovito prikupe
podatke iz različitih izvora i formata, poput baza podataka, datoteka ili
senzorskih uređaja, te ih jednostavno učitaju u Jezersko Skladište Podataka.
Ovaj sloj obično uključuje alate za obradu velikih količina podataka, poput
Apache Spark-a, kako bi se omogućilo brzo i učinkovito prikupljanje i spremanje
velikih količina podataka. Za detaljni opis Sloja Unosa Podataka vidjeti
\cite[c.2.2.1]{datalakehouse2022}.

\section{Sloj Pohrane Podataka} \label{section:sloj_pohrane_podataka}
U Jezerskom Skladištu Podataka, Sloj Pohrane Podataka obuhvaća skup tehnologija
i alata za pohranu velikih količina podataka u različitim formatima, kao što su
Apache Hadoop Distributed File System (HDFS), Amazon S3, Azure Blob Storage i
Google Cloud Storage.

Osim toga, Delta Lake tehnologija može se koristiti kao sloj pohrane podataka u
Jezerskom Skladištu, jer omogućuje verzioniranje podataka, upravljanje
transakcijama i omogućuje pohranu podataka u strukturiranom obliku, čime se
olakšava proces analize.

Sloj pohrane podataka u Jezerskom Skladištu Podataka mora biti dizajniran i
konfiguriran na način koji omogućuje brzi i jednostavan pristup podacima za
analizu, uz osiguravanje pouzdanosti, sigurnosti i skalabilnosti skladišta. Za
detaljni opis Sloja Pohrane Podataka vidjeti \cite[c.2.2.2]{datalakehouse2022}.

\section{Sloj Obrade Podataka} \label{section:sloj_obrade_podataka}
Sloj Obrade Podataka (eng. Data Processing Layer) u Jezerskom Skladištu Podataka
odnosi se na skup tehnologija i alata koji omogućuju obradu velikih količina
podataka pohranjenih u Jezerskom Skladištu Podataka. Ovaj sloj obično uključuje
distribuirane obradne okvire, poput Apache Sparka, Apache Flinka i Apache Beam-a
, te različite servise za obradu podataka kao što su Apache Hive, Apache Pig,
Apache Sqoop i Apache Storm.

Sloj Obrade Podataka je dizajniran kako bi omogućio izvođenje različitih
operacija na podacima, uključujući čišćenje, transformiranje, spajanje i
agregiranje. Ovaj sloj omogućuje korisnicima da lako i učinkovito izvode složenu
obradu velikih skupova podataka, koristeći alate za distribuiranu obradu.

Sloj Obrade Podataka u Jezerskom Skladištu Podataka igra ključnu ulogu u
omogućavanju pouzdane, skalabilne i brze obrade podataka pohranjenih u Jezerskom
Skladištu Podataka, što omogućuje korisnicima da izvuku vrijednost iz podataka i
donose informirane poslovne odluke. Za detaljni opis Sloja Obrade Podataka
vidjeti \cite[c.2.2.3]{datalakehouse2022}.