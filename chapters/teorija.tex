\chapter{Teorija}
U ovom poglavlju se obrađuje teorija Jezerskog Skladišta Podataka. U ovom
poglavlju dana je definicija Jezerskog Skladišta Podataka, te su opisani
sljedeći slojevi:
\begin{itemize}
    \item Sloj Unosa Podataka,
    \item Sloj Obrade Podataka,
    \item Sloj Pohrane Podataka.
\end{itemize}

\section{Jezersko Skladište Podataka}
Jezersko Skladište Podataka (eng. Data Lakehouse) je arhitektura skladištenja
podataka koja kombinira karakteristike Jezera Podataka (eng. Data Lake) i
Skladišta Podataka (eng. Data Warehousea). To je centralizirano skladište
podataka koje omogućava analizu velike količine strukturiranih i
nestrukturiranih vrsta podataka u realnom vremenu ili u kasnijem trenutku.
Jezersko skladište podataka omogućuje integraciju podataka iz različitih izvora,
olakšava upravljanje podacima, smanjuje troškove i vrijeme potrebno za pripremu
podataka za analizu. Ova arhitektura skladištenja podataka postaje sve
popularnija u posljednje vrijeme jer olakšava analizu podataka, izvještavanje i
donošenje odluka u stvarnom vremenu. Za detaljniji opis Jezerskog Skladišta
Podataka vidjeti \citep{datalakehouse2022}.
